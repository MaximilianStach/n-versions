
%%%%%%%%%%%%%%%%%%%%%%% file typeinst.tex %%%%%%%%%%%%%%%%%%%%%%%%%
%
% This is the LaTeX source for the instructions to authors using
% the LaTeX document class 'llncs.cls' for contributions to
% the Lecture Notes in Computer Sciences series.
% http://www.springer.com/lncs       Springer Heidelberg 2006/05/04
%
% It may be used as a template for your own input - copy it
% to a new file with a new name and use it as the basis
% for your article.
%
% NB: the document class 'llncs' has its own and detailed documentation, see
% ftp://ftp.springer.de/data/pubftp/pub/tex/latex/llncs/latex2e/llncsdoc.pdf
%
%%%%%%%%%%%%%%%%%%%%%%%%%%%%%%%%%%%%%%%%%%%%%%%%%%%%%%%%%%%%%%%%%%%


\documentclass[runningheads,a4paper]{llncs}

\usepackage{amssymb}
\setcounter{tocdepth}{3}
\usepackage{graphicx}
\usepackage[utf8]{inputenc}

\usepackage{url}
 
\newcommand{\keywords}[1]{\par\addvspace\baselineskip
\noindent\keywordname\enspace\ignorespaces#1}

\begin{document}

\mainmatter  % start of an individual contribution

% first the title is needed
\title{N-Version Programming}

% a short form should be given in case it is too long for the running head
\titlerunning{Lecture Notes in Computer Science: Authors' Instructions}


\author{Maximilian Stach}
%
\authorrunning{Lecture Notes in Computer Science: Authors' Instructions}
% (feature abused for this document to repeat the title also on left hand pages)

% the affiliations are given next; don't give your e-mail address
% unless you accept that it will be published
\institute{Fakultät Informatik, TU Dortmund
%W\url{http://.ai.cs.tu-dortmund.de}
}



\maketitle


\begin{abstract}
In dieser Arbeit werden die Konzepte und Eigenschaften des N-Version Programming kurz vorgestellt. Desweiteren werden mögliche Anwendungsbereiche für N-Version Programming in sicherheitskritischen Softwarebereichen aufgezeigt und auf mögliche Kosten und Nutzen im Vergleich zu weiteren Programmierparadigmen eingegangen.
\keywords{N-Version Programming, Sicherheit von Softwarearchitekturen, Zero-Day Exploits,...}
\end{abstract}
\tableofcontents

\cleardoublepage

\section{Einleitung} \label{einleitung}
\subsection{Motivation}\label{motivation}

Die Bedeutung von fehlerfreien und zuverlässigen Programmen in modernen Aufgabenfeldern der Software-Entwicklung ist allgegenwärtig.
Insbesondere für sicherheitskritische Bereiche, in denen falsche oder zu spät eintreffende Ergebnisse zu fatalen Folgen führen können, wird ein hohes Maß an Aufwand betrieben, um Fehler im produktiven Betrieb auszuschließen.
Detaillierte Konzeptbeschreibungen, Korrektheitsbeweise und automatisierte Testläufe der entwickelten Programme gehören zu gängigen Ansätzen um dies zu erreichen.
In den 1970er Jahren beginnend wurden auf Redundanz von Hardwarekomponenten beruhende Konzepte zur Fehlervermeidung auf die Software-Ebene übertragen und als \enquote{\emph{N-Version Programming}} \cite{Chen1978} bekannt.
Die N-Version Programmierung ist ein Ansatz der Entwicklung von Software-Archi"-tekturen um eine möglichst fehlerfreie Ausführung von Software durch Entwicklung von mindestens zwei funktional äquivalenten Software-Modulen zu erreichen, die auch Versionen genannt werden.
Sie zielt darauf ab, die verschiedenen Versionen von unabhängigen Entwicklerteams nach einer gemeinsamen Spezifikation zu entwerfen und durch die unabhängige Entwicklung gleiche Fehler in der Entwicklung der Versionen zu verhindern.
Die Ergebnisse einer Anfrage an die unabhängig von einander entworfenen Versionen werden miteinander verglichen um inkorrekte Ergebnisse auszuschließen.
Das Ziel dieser Arbeit besteht insbesondere darin, die Geschichte und Konzepte der N-Version Programmierung zu erläutern und einige bedeutende Studien vorzustellen, welche die Effektivität des Ansatzes zur Fehlervermeidung untersucht haben.


\subsection{Aufbau der Arbeit}\label{aufbau}

In dieser Arbeit wird zunächst in Kapitel \ref{geschichte} ein Überblick über die Geschichte sowie über die Konzepte der N-Version Programmierung aufgezeigt, woraufhin verschiedene Definitionen und Konzepte des Ansatzes in Kapitel \ref{konzepte} kurz beschrieben werden. Nachfolgend werden in Kapitel \ref{beispiele} Projekte vorgestellt, welche Programme nach den Prinzipien des N-Version Programming entwickelt haben und auf die besonderen Anforderungen, Schwierigkeiten und Ergebnisse, die dadurch entstanden, eingegangen.
Um Erkenntnisse bezüglich der Verwendbarkeit des Ansatzes auf aktuelle Problemstellungen der Software-Entwicklung gewinnen zu können, werden die im vorherigen Kapitel \ref{beispiele} ausgearbeiteten Beobachtungen bezüglich der Praxis von N-Version Programming in Kapitel \ref{bewertung} zusammengefasst und das Einsatzpotential für aktuelle Problemfelder der Softwareentwicklung diskutiert.
Abschließend werden die vorgestellten Konzepte und die gewonnenen Beobachtungen über die N-Version Programmierung in Kapitel \ref{zusammenfassung-ausblick} zusammengefasst und ein Ausblick über die Anwendbarkeit und Relevanz von N-Version Programmierung für zukünftige Softwareprojekte skizziert.


\section{Stand der Forschung} \label{stand}
\input{kapitel/stand.tex}


\section{Definition} \label{definition}
\subsection{Geschichte}\label{geschichte}

Design-Fehler, inkorrekte Installationen und Mutwillige Angriffe zählen mit zu den Risiken in Software-Systemen. Sie können beispielsweise zu falschen oder zu spät eintreffenden Ergebnissen, Datenverlusten und als weitere Folge zu wirtschaftlichen und persönlichen Schäden führen \cite{Laprie:1995:DCC:1899254.1899261}.
Eine typische Herangehensweise der Softwareentwicklung um Fehler in Programmen zu vermeiden, besteht in der möglichst vollständigen Eliminierung von Design-Fehlern vor der Ausführung im produktiven Betrieb \cite{Avizienis:1975:FFC:800027.808469}. 
Die Idee, durch redundante Berechnungen von Ergebnissen Fehler zu erkennen und zu vermeiden, geht lange zurück. Bereits 1834 schrieb der irische Naturwissenschaftler Dionysius Lardner: \enquote{\emph{The most certain and effectual check upon errors which arise in the process of computation,	is to cause the same computations to be made by separate and independent computers; and this	check is rendered still more decisive if they make their computations by different methods.}} \cite{lardner}.

Mit der Motivation, vor allem Laufzeitfehler, die aufgrund von Design-Fehlern auftreten, zu vermeiden, erschienen Anfang der 1970er Jahre Vorschläge zum Einsatz von mehreren funktional äquivalenten Versionen des Zielprogramms \cite{methodology}.
Eine Auslegung des Konzepts der Verwendung von Redundanz zum Erzielen von fehlertoleranten Systemen wurde bereits 1974 als \enquote{Recovery Block}-Ansatz bekannt \cite{Horning:1974:PSE:647641.733522}.
Hierbei werden Ergebnisse eines primären Blocks des Programms, wie in Abbildung \ref{graph-recovery} angedeutet, durch einen Akzeptanztest überprüft.
Falls der Akzeptanztest bei dem primären Block scheitert,wird der erste sekundäre Block ausgeführt. Dieser Prozess wird wiederholt bis ein Akzeptanztest das Ergebnis eines Blocks akzeptiert oder das System aufgrund mangelnder weiterer Blocks abbricht.
%
%
\begin{figure}[ht]
	\centering
	\includegraphics[width=0.8\textwidth,natwidth=901,natheight=333]{grafiken/recovery-block.png}
	\caption{Prinzip der Recovery-Blocks \cite{lardner}}
	\label{graph-recovery}
\end{figure}
%
%
Der zunächst als \enquote{redundant programming} bezeichnete Ansatz, einen systematischen Prozess zur Entwicklung von funktional equivalenten, multiversionalen Programmen, zu definieren, wurde später in \enquote{N-version programming} umbennant.
Wie in Abbildung \ref{graph-n-version-single} ersichtlich, liegt der wesentliche Unterschied zum Ansatz der Recovery-Blocks darin, dass die als Versionen bezeichneten Blöcke in jedem Fall alle ausgeführt werden und statt eines Akzeptanztests ein Voting auf Basis aller Ergebnisse der verschiedenen Versionen durchgeführt wird.
Die begründende Arbeit dazu veröffentlichten Liming Chen und Algirdas Avizienis 1977 mit \enquote{\emph{N-version programming: A fault-tolerance approach to reliability of software operation}} \cite{Chen1978}.
%
%
\begin{figure}[ht]
	\centering
	\includegraphics[width=0.8\textwidth,natwidth=901,natheight=333]{grafiken/single-thread-n-version.png}
	\caption{Voting Prinzip in der N-Version Programmierung bei Single threading \cite{lardner}}
	\label{graph-n-version-single}
\end{figure}
%
%

\subsection{Konzepte} \label{konzepte}

- Definition
- assumptions in the reasoning for this approach
- Process of building software
- used concepts
- mechanisms
- critical points



\section{N-Version in durchgeführten Projekten} \label{beispiele}
%
Im vorherigen Kapitel \ref{definition} wurde die Geschichte der N-Version Programmierung kurz vorgestellt und auf die zugrunde liegenden Konzepte eingegangen.
Dabei wurde festgehalten, dass die Zuverlässigkeit eines N-Version Programms wesentlich davon abhängt, wie unabhängig die verschiedenen Versionen eines Programms und wie unabhängig folglich wie unabhängig die auftretenden Fehler zur Laufzeit sind. Damit sich der zusätzliche Entwicklungsaufwand der verschiedenen Versionen lohnt, dürfen in verschiedenen Versionen nicht stets dieselben Fehler auftreten.
In den nachfolgenden Kapiteln werden Studien vorgestellt, die diese Annahme der Unabhängigkeit der Versionen und Fehler untersucht haben.
%
\subsection{Studien UCLA}\label{uclastudies}

\subsection{John Knight und Nancy Leveson}\label{matrixexperiement}



\subsection{Webserver}\label{webserver}

\section{Vergleich mit anderen Ansätzen} \label{bewertung}
%
In den vorherigen Kapiteln wurde Konzepten, Eigenschaften und Kritikpunkte der N-Version Programmierung vorgestellt.
Potentielle Stärken und Schwächen dieses und anderer Ansätze zur Zuverlässigkeit von Software-Systemen werden in Kapitel \ref{vergleich} entgegengesetzt. 
Weiterhin werden in Kapitel \ref{bewertung-relevanz} mögliche Einsatzmöglichkeiten der N-Version Programmierung für aktuelle Felder der Software-Entwicklung vorgestellt.
%
\subsection{Bewertung}\label{vergleich}
Erhöhter Aufwand bei der Entwicklung von Software im Hinblick auf fehlerfreie Ausführung ist besonders in sicherheitskritischen Problemfeldern, bei denen auftretende Fehler zu fatalen Folgen führen können, sinnvoll.
Um zuverlässige Ausführung von Programmen zu Erreichen gibt es zwei verschiedene Ansätze \cite{Avizienis:1975:FFC:800027.808469}. Eine Möglichkeit besteht in dem Bestreben möglichst alle Fehler vor dem produktiven Einsatz zu eliminieren. Intensiven Tests des Zielprogramms, standardisierten Methoden des Software-Designs und die Wahl von höheren Programmiersprachen zählen zu den Herangehensweisen. Jedoch hat sich anhand vieler fataler Fehler in der Geschichte der Programmierung gezeigt, dass trotz aller Bemühungen keine absolute Zuverlässigkeit garantiert werden kann. Das alternative Verfahren dazu liegt im Versuch Fehlertoleranz bei der Ausführung von Programmen zu erzielen. Falls Fehler auftreten, werden diese durch redundante Strukturen aufgefangen und spiegeln sich nicht im weiteren Ablauf oder in den Ergebnissen der Programme wider. N-Version Programmierung als Ansatz zur Fehlertoleranz ermöglicht in der Theorie durch die parallele Entwicklungsmöglichkeiten kürzere Entwicklungszeiten als intensiv getestete Programme, welche lediglich aus einer Version bestehen. Durch die erhöhte Anzahl der beteiligten Entwicklerteams würden hingegen auch hier höhere Kosten vorhanden sein. Die Studienlage zur Effektivität der N-Version Programmierung im Hinblick auf Kosten und Zuverlässigkeitsgewinn ist uneindeutig. Obwohl eine höhere Zuverlässigkeit durch N-Version Programmierung erzielt werden konnte, ergaben Experimente, dass die Annahme der Unabhängigkeit der auftretenden Fehler in verschiedenen Versionen nicht in jedem Fall gegeben ist \cite{Knight:1986:EEA:10677.10688} und multiversionale Programmierung nicht als Ersatz für ausgiebiges funktionales Testen herhält \cite{Shimeall:1991:ECS:104878.104899}. Außerdem wurde gezeigt, dass bereits geringe Wahrscheinlichkeiten für korrelierte Fehler in den Ergebnissen der Versionen zu einer wesentlichen Reduktion der potentiellen Zuverlässigkeitssteigerung führen \cite{Eckhardt:1985:TBA:1314034.1314066}.
Weitere Ansätze sind neben der Recovery-Block-Technik unter anderem \enquote{Concurrent Error Detection} und \enquote{Algorithmic Fault-tolerance} \cite{229487}. Alle haben gemein, dass sie für bestimmte Problemstellungen besser geeignet sind und unterschiedliche Qualitäten zur Fehlerentdeckung- und Vermeidung haben. Teilweise wird eine Balance zwischen Bestrebungen zur Fehlervermeidung- und Toleranz vorgeschlagen \cite{Avizienis:1975:FFC:800027.808469}.
%
\subsection{Aktuelle Relevanz}\label{bewertung-relevanz}
Aktuellere Studien widmen sich erneut dem Ansatz der multiversionalen Software-Entwicklung, jedoch diesmal verstärkt mit der Absicht die Sicherheit von Programmen gegen mutwillige Angriffe zu erhöhen \cite{current-challenges}. Diversität lediglich auf der Ebene der Implementation wird als nicht ausreichend eingestuft und zusätzlich verstärkt, wie durchaus ursprünglich gefordert, auf die Ebene der Programmiersprachen fokussiert. Obwohl die Hypothese der Fehlerunabhängigkeit nicht bestätigt wurde, so konnte doch eine wesentlich höhere Zuverlässigkeit durch den Einsatz von Diversität erzielt werden.
Da eine häufige Kategorie von Fehlern in multiversionalen Programmen auf Missverständnisse in der Interpretation der gemeinsamen Spezifikation zurückzuführen sind, wird zunehmend die Forderung der N-Version Programmierung nach isolierten Programmiertätigkeiten diskutiert. Wären Diskussionen unter den verschiedenen Entwicklerteams über die Interpretation der Spezifikation nicht verboten sondern erwünscht, könnte diese Art der Fehler deutlich reduzieren. Weiterhin wird gefordert, dass gezielt die Diversität der Programme durch Wahl unterschiedlicher Programmiersprachen und bereitstellen verschiedener Spezifikationen erhöht werden sollte. 
Auch werden Internetapplikationen vermehrt als Einsatzgebiet der N-Version Programmierung erforscht. Eine Cloud-basierte Antiviren-Software konnte beispielsweise durch die Verwendung mehrerer heterogener Virenerkennungssysteme eine Steigerung der Entdeckungsrate von aktuellen Viren um $35\%$ im Vergleich zu herkömmlichen Systemen verzeichnen \cite{Oberheide:2008:CNA:1496711.1496718}. Dass selbst das Generieren des HTML-Quelltextes von Websiten und das damit verbundene Backend mit N-Version Programmierung umgesetzt werden kann zeigte ein weiteres Experiment \cite{zero-day}. 
Dabei wurden 3 Versionen eines Online-Auktionshauses mit je verschiedenen Betriebsystemen, Webservern, Programmiersprachen und Datenbankmanagementsystemen implementiert. Es sollte die Sicherheit im Hinblick auf 10 typische Schwachstellen von Web-Applikationen, die besonders von Zero-Day-Exploits geprägt sind, verbessert werden. Da Zero-Day-Exploits meist plattform- oder programmiersprachenspezifisch sind, ist anzunehmen, dass sie nur in einer der 3 Versionen auftreten. Schwachstelle dabei bleibt jedoch der notwendige gemeinsame Verteiler und Treiber der HTTP-Anfragen. Dieser muss zusätzlich zum Verteilen der Anfragen die Session des Benutzers mit der Website verwalten und fehlgeschlagene Anfragen protokollieren.
Dieser Verteiler kann selbst zum Ziel von Attacken werden und falls kompromittiert, Transaktionen im Namen eines anderen Benutzers tätigen.

\section{Zusammenfassung und Ausblick} \label{zusammenfassung-ausblick}
%
In dieser Arbeit wurde zunächst die Geschichte der N-Version Programmierung kurz vorgestellt.
Sie ist ein Ansatz zur Entwicklung fehlertoleranter Softwaresysteme und basiert darauf, mehrere Versionen des Zielprogramms einer gemeinsamen Spezifikation zu entwickeln und die Ergebnisse der einzelnen Versionen des Programms zur Laufzeit miteinander zu vergleichen.
Es wird angenommen, dass in Ergebnissen einzelner Versionen aufgetretene Fehler durch Diversität in der Entwicklung der Versionen unabhängig voneinander sind und so trotz eines Fehler in einer Version stets das korrekte Ergebnis berechnet wird.
Die benötigten besonderen Komponenten zum Entwurf von N-Version Software wurden beschrieben.
Eine Studie, welche die Eignung unterschiedlicher Spezifikationssprachen untersuchte, wurde vorgestellt und ein Experiment, welches die Annahme über die Fehlerunabhängigkeit überprüft hat wurde diskutiert. Es wurde herausgestellt, dass Unstimmigkeit bei den Studien zur Effektivität der N-Version Programmierung besteht und weitere Experimente notwendig sind um die vorhandenen Ansätze zur Fehlertoleranz in Software-Systemen auf Anwendbarkeit, Effektivität und Aufwand zu untersuchen.
Häufige Ursache von korreliert aufgetretenen Fehlern lag in Missinterpretationen der Spezifikation und mangelndem mathematischen Hintergrundwissen. Der versprochene verringerte Testaufwand bei Beibehaltung hoher Zuverlässigkeit konnte durch N-Version Programmierung nicht verlässlich erzielt werden.
Weiterhin wurden aktuelle Arbeiten zur Relevanz der N-Version Programmierung in Internetdiensten vorgestellt. Der Einsatz von multiplen Programmiersprachen und Laufzeitumgebungen in den einzelnen Versionen könnte insbesondere Anfälligkeiten aufgrund von Zero-Day-Exploits verhindern, da diese meist plattformspezifisch sind.








\begin{thebibliography}{4}

\bibitem{jour} Smith, T.F., Waterman, M.S.: Identification of Common Molecular
Subsequences. J. Mol. Biol. 147, 195--197 (1981)

\bibitem{lncschap} May, P., Ehrlich, H.C., Steinke, T.: ZIB Structure Prediction Pipeline:
Composing a Complex Biological Workflow through Web Services. In: Nagel,
W.E., Walter, W.V., Lehner, W. (eds.) Euro-Par 2006. LNCS, vol. 4128,
pp. 1148--1158. Springer, Heidelberg (2006)

\bibitem{book} Foster, I., Kesselman, C.: The Grid: Blueprint for a New Computing
Infrastructure. Morgan Kaufmann, San Francisco (1999)

\bibitem{proceeding1} Czajkowski, K., Fitzgerald, S., Foster, I., Kesselman, C.: Grid
Information Services for Distributed Resource Sharing. In: 10th IEEE
International Symposium on High Performance Distributed Computing, pp.
181--184. IEEE Press, New York (2001)

\bibitem{proceeding2} Foster, I., Kesselman, C., Nick, J., Tuecke, S.: The Physiology of the
Grid: an Open Grid Services Architecture for Distributed Systems
Integration. Technical report, Global Grid Forum (2002)

\bibitem{url} National Center for Biotechnology Information, \url{http://www.ncbi.nlm.nih.gov}

\end{thebibliography}


\section*{Anhang}
\input{kapitel/anhang.tex}

\end{document}
