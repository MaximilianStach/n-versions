
%%%%%%%%%%%%%%%%%%%%%%% file typeinst.tex %%%%%%%%%%%%%%%%%%%%%%%%%
%
% This is the LaTeX source for the instructions to authors using
% the LaTeX document class 'llncs.cls' for contributions to
% the Lecture Notes in Computer Sciences series.
% http://www.springer.com/lncs       Springer Heidelberg 2006/05/04
%
% It may be used as a template for your own input - copy it
% to a new file with a new name and use it as the basis
% for your article.
%
% NB: the document class 'llncs' has its own and detailed documentation, see
% ftp://ftp.springer.de/data/pubftp/pub/tex/latex/llncs/latex2e/llncsdoc.pdf
%
%%%%%%%%%%%%%%%%%%%%%%%%%%%%%%%%%%%%%%%%%%%%%%%%%%%%%%%%%%%%%%%%%%%


\documentclass{llncs}

\usepackage[german,english]{babel}
\usepackage{amssymb}
\setcounter{tocdepth}{3}
\usepackage{graphicx}
\usepackage[utf8]{inputenc}
\usepackage{csquotes}

\usepackage{url}
 
\newcommand{\keywords}[1]{\par\addvspace\baselineskip
\noindent\keywordname\enspace\ignorespaces#1}

\begin{document}
\selectlanguage{german}
\mainmatter  % start of an individual contribution

% first the title is needed
\title{N-Version Programmierung}

% a short form should be given in case it is too long for the running head
\titlerunning{N-Version Programmierung}


\author{Maximilian Stach}
%
\authorrunning{Maximilian Stach}
% (feature abused for this document to repeat the title also on left hand pages)

% the affiliations are given next; don't give your e-mail address
% unless you accept that it will be published
\institute{Fakultät Informatik, TU Dortmund
%W\url{http://.ai.cs.tu-dortmund.de}
}



\maketitle


\begin{abstract}
Beim Ansatz der N-Version Programmierung werden mehrere funktional äquivalente Versionen eines Software-Moduls erstellt und ihre Ergebnisse zur Laufzeit miteinander verglichen um aufgetretene Fehler zu erkennen.
In dieser Arbeit werden die Geschichte und Konzepte der N-Version Programmierung kurz dargestellt. Weiterhin werden dazu durchgeführte Studien vorgestellt und auf die aktuelle Relevanz und Einsatzmöglichkeit in Software-Projekten eingegangen.

\keywords{N-Version Programmierung, Redundanz, Sicherheit von Softwarearchitekturen, Fehlertoleranz}

\end{abstract}

\tableofcontents

\cleardoublepage

\section{Einleitung} \label{einleitung}
\subsection{Motivation}\label{motivation}

Die Bedeutung von fehlerfreien und zuverlässigen Programmen in modernen Aufgabenfeldern der Software-Entwicklung ist allgegenwärtig.
Insbesondere für sicherheitskritische Bereiche, in denen falsche oder zu spät eintreffende Ergebnisse zu fatalen Folgen führen können, wird ein hohes Maß an Aufwand betrieben, um Fehler im produktiven Betrieb auszuschließen oder zumindest zu tolerieren.
Detaillierte Konzeptbeschreibungen, Korrektheitsbeweise und automatisierte Testläufe der entwickelten Programme gehören zu gängigen Ansätzen um dies zu erreichen.
In den 1970er Jahren beginnend wurden auf Redundanz von Hardwarekomponenten beruhende Konzepte zur Fehlervermeidung auf die Software-Ebene übertragen und eine Variante davon als \enquote{\emph{N-Version Programming}} \cite{Chen1978} bekannt.
Die N-Version Programmierung ist ein Ansatz der Entwicklung von Software-Archi"-tekturen um eine möglichst fehlerfreie Ausführung von Software durch Entwicklung von mindestens zwei funktional äquivalenten Software-Modulen zu erreichen, die auch Versionen genannt werden.
Sie zielt darauf ab, die verschiedenen Versionen von unabhängigen Entwicklerteams nach einer gemeinsamen Spezifikation zu entwerfen und durch die unabhängige Entwicklung identische Fehler in der Entwicklung der Versionen zu verhindern.
Die Ergebnisse einer Anfrage an die unabhängig von einander entworfenen Versionen werden miteinander verglichen um inkorrekte Ergebnisse auszuschließen.
Das Ziel dieser Arbeit besteht insbesondere darin, die Geschichte und Konzepte der N-Version Programmierung zu erläutern und einige bedeutende Studien vorzustellen, welche die Effektivität des Ansatzes zur Fehlertoleranz untersucht haben.


\subsection{Aufbau der Arbeit}\label{aufbau}

In dieser Arbeit wird zunächst in Kapitel \ref{geschichte} ein Überblick über die Geschichte sowie über die Konzepte der N-Version Programmierung aufgezeigt, woraufhin verschiedene Definitionen und Konzepte des Ansatzes in Kapitel \ref{konzepte} kurz beschrieben werden. Nachfolgend werden in Kapitel \ref{beispiele} Projekte vorgestellt, welche Programme nach den Prinzipien des N-Version Programming entwickelt haben und auf die besonderen Anforderungen, Schwierigkeiten und Ergebnisse, die dadurch entstanden, eingegangen.
Um Erkenntnisse bezüglich der Verwendbarkeit des Ansatzes auf aktuelle Problemstellungen der Software-Entwicklung gewinnen zu können, werden die im vorherigen Kapitel \ref{beispiele} ausgearbeiteten Beobachtungen bezüglich der Effektivität zur Fehlertoleranz von N-Version Programming in Kapitel \ref{bewertung} zusammengefasst, verschiedene Ansätze zum Erzielen zuverlässiger Software vorgestellt und das Einsatzpotential für aktuelle Problemfelder der Softwareentwicklung diskutiert.
Abschließend werden die vorgestellten Konzepte und die gewonnenen Beobachtungen über die N-Version Programmierung in Kapitel \ref{zusammenfassung-ausblick} zusammengefasst und ein kurzer Ausblick über die Anwendbarkeit und Relevanz von N-Version Programmierung für zukünftige Softwareprojekte skizziert.

\section{N-Version Programmierung} \label{definition}
\subsection{Geschichte}\label{geschichte}

Ein Teilaspekt der modernen Softwareentwicklung befasst sich mit dem Auffinden und vorzeitigen Vermeiden von Designfehlern in zu entwickelnder Software. Mit diesem Ziel als Motivation erschienen Anfang der 70er Vorschläge zum Einsatz von mehreren Versionen des Zielprogramms. 

Das Konzept der Verwendung von Redundanz zum Erzielen von fehlertoleranten Systemen war bereits vorallem auf Hardwareebene als \enquote{recovery block} Ansatz bekannt.

-- Kurzbeschreibung Recovery blocks, dynamic redundancy(standby sparing)

Der zunächst als \enquote{redundant programming} bezeichnete Ansatz, einen systematischen Prozess zur Entwicklung von funktional equivalenten, multiversionalen Programmen, zu definieren, wurde später in \enquote{N-version programming} umbennant.
Die begründende Arbeit dazu veröffentlichten Liming Chen und Algirdas Avizienis 1977 mit "N-version programming: A fault-tolerance approach to reliability of software operation".
%
%
%

%%%

\subsection{Konzepte}

- Definition
- assumptions in the reasoning for this approach
- Process of building software
- used concepts
- mechanisms
- critical points



\section{Forschungsprojekte} \label{beispiele}
%
Im vorherigen Kapitel \ref{definition} wurde die Geschichte der N-Version Programmierung kurz vorgestellt und auf die zugrunde liegenden Konzepte eingegangen.
Dabei wurde festgehalten, dass die Zuverlässigkeit eines N-Version Programms wesentlich davon abhängt, wie unabhängig die verschiedenen Versionen eines Programms und wie unabhängig folglich die auftretenden Fehler zur Laufzeit sind. Damit sich der zusätzliche Entwicklungsaufwand der verschiedenen Versionen lohnt, dürfen in den einzelnen Versionen nicht stets dieselben Fehler auftreten.
Zudem stellt die Spezifikation eine entscheidende Komponente dar, welche möglichst eindeutig, vollständig und korrekt zu sein hat.
In den nachfolgenden Kapiteln werden Studien vorgestellt, die diese Annahme der Unabhängigkeit der auftretenden Fehler und die Unterschiede bezüglich der Spezifikationssprachen untersucht.
%
\subsection{Wahl der Spezifikationssprache}\label{uclastudies}
Die University of California, Los Angeles (UCLA) beschäftigte sich ab 1975 unter anderem damit, welche Anforderungen an die Spezifikationen zu stellen sind, welche Problemfelder zum Einsatz von N-Version Programmen geeignet sind und mit welchen Methoden sich die Effektivität in Anbetracht der Kosten zu alternativen Ansätzen vergleichen lässt \cite{Avizienis:1985:NAF:1314034.1314064}.
In einer Studie der UCLA von 1978 - 1983 wurde anhand drei verschiedener Spezifikationssprachen untersucht, welche Auswirkungen die Wahl der Sprachen auf die zu entwickelnde multiversionale Software hat \cite{methodology}.
\subsubsection{Experimentaufbau}\label{ucla-experiment}{
Als algebraische Spezifikationssprache wurde \emph{OBJ} ausgewählt. Zusätzlich wurde die in der Industrie weit verbreitete Sprache \emph{PDL} und als Kontrollsprache Englisch untersucht.
Die gemeinsame Aufgabe bestand darin, ein Programm zum Verwalten von Flügen eines Flughafens zu schreiben, welches das Scheduling der Flüge inklusive Sitzplatzbelegungen implementiert.
Da die Aufgabenstellung transaktionsorientiert war, fiel die Wahl der Vergleichsstellen im Ablauf der Versionen auf die Transaktionsendpunkte.
An dem Experiment waren 30 Programmierer beteiligt, die insgesamt 18 Programmversionen in der Programmiersprache PL/1 erstellten.
Davon basierten 7 auf OBJ, 5 auf PDL und 6 auf der englischen Spezifikation.
Mit 74 Seiten war die PDL-Spezifikation deutlich länger als die englische mit 10 und die OBJ-Spezifikation mit 13 Seiten.
Die 18 Versionen wurden zunächst einmal gemeinsam für 100 Testtransaktionen durchgeführt und ihre Ergebnisse mit einem 18-fachen Vergleichsalgorithmus ausgewertet, um die erwarteten Resultate der Transaktionen zu erhalten.
Anschließend wurden alle 816 möglichen Triple der 18 Versionen mit einem 3-fachen Vergleichsalgorithmus auf den Testtransaktionen ausgeführt.
%
\subsubsection{Ergebnisse}
%
Die erste Beobachtung lag darin, dass es es deutliche Unterschiede in der Größe und Laufzeit der unterschiedlichen Versionen gab.
Bei Ausführung aller Testtransaktionen mit 3-fachem Vergleichsalgorithmus erzielten $80.1\%$ der Vergleichsergebnisse über alle Tripel ein semantisch korrektes Ergebnis.
In den restlichen Fällen war entweder das Ergebnis falsch oder alle drei Versionen brachen bei der Ausführung mit einer Fehlerbehandlung ab.
Triple aus OBJ-Versionen erzielten $78.2\%$, aus PDL-Versionen $85.8\%$, aus englischen Versionen $78.6\%$ und aus gemischten Versionen $80.9\%$ semantisch korrekte Ergebnisse. 
Dies zeigt, dass sich der zusätzliche Aufwand beim Erstellen der PDL-Spezifikation in besseren Ergebnissen wieder spiegelt.
Weiterhin wurden die aufgetretenen unentdeckten Fehler genauer auf ihre Ursache untersucht und in die Kategorien \emph{Spezifikationsfehler}, \emph{Interpretationsfehler} und \emph{Implementationsfehler} eingeteilt.
Fehler in der Spezifikation traten alleinig in der OBJ- und in der Englisch-Spezifikation auf.
Vor allem ausbleibende Eingabeüberprüfungen, welche zu Programmabbrüchen führten, gehörten zu den Interpretationsfehlern, die am stärksten in den PDL-Versionen vertreten waren.
Der Großteil der Implementationsfehler wurden mit dem Umstand mangelnder Testzeit erklärt.
Jedoch bestand weiterhin die Frage, ob sich der höhere Aufwand, welcher in höheren Kosten resultiert bei der erwarteten Steigerung der Fehlertoleranz lohnt.

Beobachtet wurde zudem, dass unabhängige Fehlerursachen in den einzelnen Versionen häufig zu den selben falschen Ergebnissen führten, falls die Endergebnisse in simplen Bestätigungsaussagen über den Erfolg einer Transaktion lagen. Ein Ansatz dieses Problem zu lösen liegt darin, komplexere Zustandsinformationen in die Endergebnisse einzubinden.
Als weitere Hürde für den Einsatz der N-Version Programmierung wurden fehlende automatisierte Werkzeuge, mangelnde Erfahrung mit formalen Spezifikationssprachen und im Allgemeinen mangelnde Präzision und Ausdrucksmöglichkeiten der untersuchten Spezifikationssprachen genannt.
Abschließend wurde festgehalten, dass keine der drei Sprachen alle Anforderungen im Hinblick auf die N-Version Programmierung erfüllt.
Zudem war die Anzahl der getesteten Versionen mit 18 relativ gering für ein solches Experiment, wenn betrachtet wird, dass lediglich 18 der beteiligten 30 Programmierer eine testbare Version entwickelten und diese größtenteils unerfahren in der Entwicklung von Programmen anhand von OBJ und PDL waren.
%
\subsection{Knight \& Leveson Experiment}\label{matrixexperiement}
%
Die Funktionsweise der N-Version Programmierung basiert auf der Annahme, dass unabhängig voneinander entwickelte Versionen unabhängige Fehler produzieren.
Ist diese Annahme korrekt, so liegt die Zuverlässigkeit des Gesamtsystems deutlich höher als die der einzelnen Versionen.
In dem Experiment von John Knight und Nancy Leveson, welches in einer ausführlichen Beschreibung 1986 veröffentlicht wurde \cite{Knight:1986:EEA:10677.10688}, sollte untersucht werden, ob die Annahme der Unabhängigkeit von aufgetretenen Fehlern in verschiedenen Versionen zutrifft. 
Dazu sollte eine große Anzahl von Versionen eines Zielprogramms entwickelt werden und durch realistische Testfälle überprüft werden, ob ein gemeinsames Versagen von zwei oder mehr Versionen häufiger auftritt, als durch die Unabhängigkeitsvermutung anzunehmen wäre.
%
%
\subsubsection{Experimentaufbau}\label{knight-aufbau}
Als zu implementierende Funktionalität der Versionen wurde die Aufgabe gestellt, aus Eingabevektoren von Radarreflexionen und gegebenen Randbedingungen zu überprüfen, ob die Reflexionen von einer eintreffenden Gefahr, wie zum Beispiel einer feindlichen Rakete, stammen und folglich eine Abfangrakete gestartet werden soll. Das auch als \enquote{\emph{launch interceptor problem}} bekannte Problem wurde bereits in weiteren Experimenten der Software-Entwicklung eingesetzt und als realistische Aufgabenstellung im Bereich Fehlertoleranz eingestuft. Außerdem lag ein sorgfältig getestetes \emph{Gold-Programm} aus früheren Experimente vor, welches als Vergleichsmaß angesetzt werden konnte.
Insgesamt wurden $27$ Versionen des Zielprogramms in der Programmiersprache \emph{Pascal} von $23$ Studenten und $4$ Doktoranden aus einer gemeinsamen Spezifikation entwickelt.
Dabei stammten $18$ der Versionen von Programmierern der University of California in Irvine und $9$ von der University of Virginia.
Neben der der boolschen Aussage, ob eine Abfangrakete gestartet werden soll, wurde eine $15 \times 15$ boolsche Matrix und ein 15-elementiger boolscher Vektor mit Zwischenbedingungen als Ergebnisse der Versionen erwartet. Ein Versagen einer Version in einem Testlaufs bestand bereits sobald eins der resultierenden $241$ Bits vom erwarteten Ergebnis abwich. Zum Entwickeln der Versionen erhielt jeder Programmierer $15$ Sätze von Eingabe- und zugehörigen Ausgabewerten.
Nachdem alle Versionen einen Abnahmetest bestanden hatten, wurden diese mit einer Millionen automatisch generierter Eingaben auf auftretende Fehler getestet.
%
\subsubsection{Ergebnisse}
%
Alle Versionen erzielten mit über $99\%$ eine hohe Zuverlässigkeit und $6$ Versionen waren gänzlich fehlerfrei. Trotzdem versagten in 1255 Testfällen mehr als eine Version.
In $2$ Fällen versagten sogar $8$ Versionen. Ein Großteil der korreliert aufgetretenen Fehler lag in falschen mathematischen Annahmen zur Berechnung von Zwischenergebnissen und konnte nicht auf die Spezifikation zurückgeführt werden.
Zur stochastischen Untersuchung der Annahme über die Unabhäng"-igkeit von aufgetretenen Fehlern in den einzelnen Versionen wurde eine Fehlerunabhängig"-keitshypothese aufgestellt.
Sie beruht auf der Definition von Unabhängigkeit für das Eintreten von Ereignissen, welche für zwei Ereignisse besagt, dass das Eintreten des einen Ereignisses die Wahrscheinlichkeit des Eintretens des anderen Ereignisses nicht beeinflusst. Im Kontext von $N$ Versionen eines Programms gilt somit für die Wahrscheinlichkeit $P$ des Versagens von Versionen  $V_{1}, V_{2}$ bei einer Eingabe stets: $P(V_{1}\given V_{2}) = P(V_{1}) \land P(V_{2}\given V_{1}) = P(V_{2})$.
Die Hypothese der Fehlerunabhängigkeit wurde durch die Testergebnisse des Experiments mit einer Konfidenz von über $99\%$ zurückgewiesen. Es traten deutlich mehr korrelierte Fehler auf, als bei einer Normalverteilung von unabhängigen Fehlern anzunehmen wäre. Es wurde jedoch festgehalten, dass die Zurückweisung der Annahme über die Unabhängigkeit von Versagen in $N$ Versionen eines Programms nicht ohne Weiteres auf andere Probleme übertragen lässt.
Die Zurückweisung einer zentralen Annahme der N-Version Programmierung und das Experiment an sich rief heftige Kritik hervor, sodass Knight und Leveson 1990 eine Antwort \cite{reply_critics} dazu veröffentlichten.
Kritisiert wurde unter anderem, dass die genauen Details des Prozesses der Entwicklung von N-Version Programmen nicht eingehalten wurden, der geringe Umfang des Experimentes  und die limitierte Diversität der entwickelten Versionen. Dieser Kritik entgegneten Knight und Leveson mit Gegenüberstellung ihres Experiments mit bisherigen Experimenten. Zudem weisen sie darauf hin, dass der entscheidende Faktor für Fehlertoleranz in Softwaresystemen nicht in der Unabhängigkeit von identischen Fehlerursachen in den Versionen sondern in der Unabhängigkeit von Fehlern bei identischen Eingaben für verschiedene Versionen besteht.

\section{Bewertung und aktuelle Relevanz} \label{bewertung}


\subsection{Effektivität der Fehlervermeidung}\label{bewertung-potential}
Potential für unterschiedliche Versionen sollte durch die Spezifikation gegeben sein 

\subsection{Kosten in der Entwicklungsphase}\label{bewertung-kosten}

\subsection{Einsatzpotential für aktuelle Problemfelder}\label{bewertung-relevanz}

\section{Zusammenfassung und Ausblick} \label{zusammenfassung-ausblick}
%
In dieser Arbeit wurde zunächst die Geschichte der N-Version Programmierung kurz vorgestellt.
Sie ist ein Ansatz zur Entwicklung fehlertoleranter Softwaresysteme und basiert darauf, mehrere Versionen des Zielprogramms eines gemeinsamen Spezifikation zu entwickeln und die Ergebnisse der einzelnen Versionen des Programms zur Laufzeit miteinander zu vergleichen.
Es wird angenommen, dass in Ergebnissen einzelner Versionen aufgetretene Fehler durch Diversität in der Entwicklung der Versionen unabhängig voneinander sind und so trotz eines Fehler in einer Version stets das korrekte Ergebnis berechnet wird.
Die benötigten besonderen Komponenten zum Entwurf von N-Version Software wurden beschrieben.
Eine Studie, welche die Eignung unterschiedlicher Spezifikationssprachen untersuchte, wurde vorgestellt und ein Experiment, welches die Annahme über die Fehlerunabhängigkeit überprüft hat wurde diskutiert. Es wurde herausgestellt, dass Unstimmigkeit bei den Studien zur Effektivität der N-Version Programmierung besteht und weitere Experimente notwendig sind um die vorhandenen Ansätze zur Fehlertoleranz in Software-Systemen auf Anwendbarkeit, Effektivität und Aufwand zu untersuchen.
Häufige Ursache von korreliert aufgetretenen Fehlern lag in Missinterpretationen der Spezifikation und mangelndem mathematischen Hintergrundwissen. Der versprochene verringerte Testaufwand bei Beibehaltung hoher Zuverlässigkeit konnte durch N-Version Programmierung nicht verlässlich erzielt werden.
Weiterhin wurden aktuelle Arbeiten zur Relevanz der N-Version Programmierung in Internetdiensten vorgestellt. Der Einsatz von multiplen Programmiersprachen und Laufzeitumgebungen in den einzelnen Versionen könnte insbesondere Anfälligkeiten aufgrund von Zero-Day-Exploits verhindern, da diese meist plattformspezifisch sind.








\bibliographystyle{splncs}
\bibliography{quellen}

%\section*{Anhang}
%Anhang text

\end{document}
