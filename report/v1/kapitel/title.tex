% first the title is needed
\title{N-Version Programmierung}

% a short form should be given in case it is too long for the running head
\titlerunning{Lecture Notes in Computer Science: Authors' Instructions}


\author{Maximilian Stach}
%
\authorrunning{Lecture Notes in Computer Science: Authors' Instructions}
% (feature abused for this document to repeat the title also on left hand pages)

% the affiliations are given next; don't give your e-mail address
% unless you accept that it will be published
\institute{Fakultät Informatik, TU Dortmund
%W\url{http://.ai.cs.tu-dortmund.de}
}



\maketitle


\begin{abstract}
Beim Ansatz der N-Version Programmierung werden mehrere funktional equivalente Versionen eines Software-Moduls erstellt und ihre Ergebnisse zur Laufzeit miteinander verglichen um aufgetretene Fehler zu erkennen.
In dieser Arbeit werden die Geschichte und Konzepte der N-Version Programmierung kurz dargestellt. Desweiteren werden dazu durchgeführte Studien vorgestellt und auf die aktuelle Relevanz und Einsatzmöglichkeit in Software-Projekten eingegangen.

\keywords{N-Version Programmierung, N-Version Programming, Redundanz,  Sicherheit von Softwarearchitekturen, Zero-Day Exploits,...}

\end{abstract}
