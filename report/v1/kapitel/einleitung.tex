Die Bedeutung von fehlerfreien und zuverlässigen Programme in modernen Aufgabenfeldern der Software-Entwicklung ist allgegenwärtig.
Insbesondere für sicherheitskritische Bereiche, in denen falsche oder zu spät eintrefende Ergebnisse zu fatalen Folgen führen können, wird ein hohes Maß an Aufwand betrieben, um Fehler zu Laufzeit auszuschließen.
Detaillierte Konzeptbeschreibungen, Korrektheitsbeweise und automatisierte Test der entwickelten Programme gehören zu gängigen Ansätzen um dies zu erreichen ---missing ref----.
In den 70er Jahren beginnend wurden auf Redundanz von Kompnenten beruhende Konzepte zur Fehlervermeidung in Hardware auf die Software-Ebene übertragen und als N-Version Programming bekannt.
N-Version Programming ist ein Ansatz der Entwicklung von Software-Architekturen um eine möglichst fehlerfreie Ausführung von Programmen durch Entwicklung von mindestens zwei funktional equivalenten Programmen zu erreichen.
Die Ergebnisse einer Anfrage and die unabhängig von einander entworfenen Programme, welche auch als Versionen bezeichnet werden, werden mit einander verglichen um inkorrekte Ergebnisse auszuschließen.

---- weitere kurze Beschreibung -----


In dieser Arbeit wird zunächst \ref{stand} ein Überblick über die Geschichte sowie über den aktuellen Stand der Forschung aufgezeigt, woraufhin verschiedene Definitionen und Konzepte des N-Version Programming in Kapitel \ref{definition} detailliert beschrieben werden.