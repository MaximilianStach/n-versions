%
Im vorherigen Kapitel \ref{definition} wurde die Geschichte der N-Version Programmierung kurz vorgestellt und auf die zugrunde liegenden Konzepte eingegangen.
Dabei wurde festgehalten, dass die Zuverlässigkeit eines N-Version Programms wesentlich davon abhängt, wie unabhängig die verschiedenen Versionen eines Programms und wie unabhängig folglich wie unabhängig die auftretenden Fehler zur Laufzeit sind. Damit sich der zusätzliche Entwicklungsaufwand der verschiedenen Versionen lohnt, dürfen in verschiedenen Versionen nicht stets dieselben Fehler auftreten.
In den nachfolgenden Kapiteln werden Studien vorgestellt, die diese Annahme der Unabhängigkeit der Versionen und Fehler untersucht haben.
%
\subsection{Studien UCLA}\label{uclastudies}

\subsection{John Knight und Nancy Leveson}\label{matrixexperiement}



\subsection{Webserver}\label{webserver}