\subsection{Geschichte}\label{geschichte}

Ein Teilaspekt der modernen Softwareentwicklung befasst sich mit dem Auffinden und vorzeitigen Vermeiden von Designfehlern in zu entwickelnder Software. Mit diesem Ziel als Motivation erschienen Anfang der 70er Vorschläge zum Einsatz von mehreren Versionen des Zielprogramms. 

Das Konzept der Verwendung von Redundanz zum Erzielen von fehlertoleranten Systemen war bereits vorallem auf Hardwareebene als \enquote{recovery block} Ansatz bekannt.

-- Kurzbeschreibung Recovery blocks, dynamic redundancy(standby sparing)

Der zunächst als \enquote{redundant programming} bezeichnete Ansatz, einen systematischen Prozess zur Entwicklung von funktional equivalenten, multiversionalen Programmen, zu definieren, wurde später in \enquote{N-version programming} umbennant.
Die begründende Arbeit dazu veröffentlichten Liming Chen und Algirdas Avizienis 1977 mit "N-version programming: A fault-tolerance approach to reliability of software operation".
%
%
%

%%%

\subsection{Konzepte}

- Definition
- assumptions in the reasoning for this approach
- Process of building software
- used concepts
- mechanisms
- critical points

