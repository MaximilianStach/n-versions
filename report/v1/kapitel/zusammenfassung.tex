%
In dieser Arbeit wurde zunächst die Geschichte der N-Version Programmierung kurz vorgestellt.
Sie ist ein Ansatz zur Entwicklung fehlertoleranter Softwaresysteme und basiert darauf, mehrere Versionen des Zielprogramms eines gemeinsamen Spezifikation zu entwickeln und die Ergebnisse der einzelnen Versionen des Programms zur Laufzeit miteinander zu vergleichen.
Es wird angenommen, dass in Ergebnissen einzelner Versionen aufgetretene Fehler durch Diversität in der Entwicklung der Versionen unabhängig voneinander sind und so trotz eines Fehler in einer Version trotzdem das korrekte Ergebnis berechnet wird.
Die benötigten besonderen Komponenten zum Entwurf von N-Version Software wurden beschrieben
Eine Studie, welche die Eignung unterschiedlicher Spezifikationssprachen untersuchte wurde vorgestellt und ein Experiment, welches die Annahme über die Fehlerunabhängigkeit überprüft hat wurde diskutiert. Es wurde herausgestellt, dass Unstimmigkeit bei den Studien zur Effektivität der N-Version Programmierung besteht und weitere Experimente notwendig sind um die vorhandenen Ansätze zur Fehlertoleranz in Software-Systemen auf Anwendbarkeit, Effektivität und Aufwand zu untersuchen.
Häufige Ursache von korreliert aufgetretenen Fehlern lag in Missinterpretationen der Spezifikation und mangelndem mathematischen Hintergrundwissen. Der versprochene verringerte Testaufwand um ein hohes Maß an Zuverlässigkeit zu erreichen konnte durch N-Version Programmierung nicht verlässlich erzielt werden.
Weiterhin wurden aktuelle Arbeiten zur Relevanz der N-Version Programmierung in Internetdiensten vorgestellt. Der Einsatz von multiplen Programmiersprachen und Laufzeitumgebungen in den einzelnen Versionen könnte insbesondere Anfälligkeiten aufgrund von Zero-Day-Exploits verhindern, da diese meist plattformspezifisch sind.



