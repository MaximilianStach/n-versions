%%%%%%%%%%%%%%%%%%%%%%%%%%%%%%%%%%%%%%%%%
% Beamer Presentation
% LaTeX Template
% Version 1.0 (10/11/12)
%
% This template has been downloaded from:
% http://www.LaTeXTemplates.com
%
% License:
% CC BY-NC-SA 3.0 (http://creativecommons.org/licenses/by-nc-sa/3.0/)
%
%%%%%%%%%%%%%%%%%%%%%%%%%%%%%%%%%%%%%%%%%

%----------------------------------------------------------------------------------------
%	PACKAGES AND THEMES
%----------------------------------------------------------------------------------------


\documentclass{beamer}
\usepackage[utf8]{inputenc} % Für Umlaute unter Linux
\definecolor{TUGreen}{HTML}{84b819} % TU-Grün definieren
\mode<presentation> {

% The Beamer class comes with a number of default slide themes
% which change the colors and layouts of slides. Below this is a list
% of all the themes, uncomment each in turn to see what they look like.

%\usetheme{default}
%\usetheme{AnnArbor}
%\usetheme{Antibes}
%\usetheme{Bergen}
%\usetheme{Berkeley}
%\usetheme{Berlin}
%\usetheme{Boadilla}
%\usetheme{CambridgeUS}
%\usetheme{Copenhagen}
%\usetheme{Darmstadt}
%\usetheme{Dresden}
%\usetheme{Frankfurt}
%\usetheme{Goettingen}
%\usetheme{Hannover}
%\usetheme{Ilmenau}
%\usetheme{JuanLesPins}
%\usetheme{Luebeck}
\usetheme{Madrid}
%\usetheme{Malmoe}
%\usetheme{Marburg}
%\usetheme{Montpellier}
%\usetheme{PaloAlto}
%\usetheme{Pittsburgh}
%\usetheme{Rochester}
%\usetheme{Singapore}
%\usetheme{Szeged}
%\usetheme{Warsaw}

% As well as themes, the Beamer class has a number of color themes
% for any slide theme. Uncomment each of these in turn to see how it
% changes the colors of your current slide theme.

%\usecolortheme{albatross}
%\usecolortheme{beaver}
%\usecolortheme{beetle}
%\usecolortheme{crane}
%\usecolortheme{dolphin}
%\usecolortheme{dove}
%\usecolortheme{fly}
%\usecolortheme{lily}
%\usecolortheme{orchid}
%\usecolortheme{rose}
%\usecolortheme{seagull}
%\usecolortheme{seahorse}
%\usecolortheme{whale}
%\usecolortheme{wolverine}

%\setbeamertemplate{footline} % To remove the footer line in all slides uncomment this line
%\setbeamertemplate{footline}[page number] % To replace the footer line in all slides with a simple slide count uncomment this line

%\setbeamertemplate{navigation symbols}{} % To remove the navigation symbols from the bottom of all slides uncomment this line
\setbeamercolor{structure}{fg=TUGreen!90!black}
}

\usepackage{graphicx} % Allows including images
\usepackage{booktabs} % Allows the use of \toprule, \midrule and \bottomrule in tables
\usepackage{amsmath}
\usepackage{listings}
\usepackage{pgf, tikz} % fuer Graphen
\usepackage[ngerman]{babel}
\usetikzlibrary{shapes,arrows} % fuer Graphen Elemente
\usepackage{csquotes}
\usepackage{url}
% Declare arg max
\DeclareMathOperator*{\argmax}{arg\,max}
\usepackage{dirtytalk}
%----------------------------------------------------------------------------------------
%	TITLE PAGE
%----------------------------------------------------------------------------------------

\title[N-Version Programmierung]{N-Version Programmierung} % The short title appears at the bottom of every slide, the full title is only on the title page

\author{Maximilian Stach} % Your name
\institute[TU Dortmund] % Your institution as it will appear on the bottom of every slide, may be shorthand to save space
{
Technische Universität Dortmund \\ % Your institution for the title page
\medskip
}
\date{14.02.2017} % Date, can be changed to a custom date

\begin{document}

\begin{frame}
\titlepage % Print the title page as the first slide
\end{frame}

\begin{frame}
\frametitle{Überblick} % Table of contents slide, comment this block out to remove it
\tableofcontents % Throughout your presentation, if you choose to use \section{} and \subsection{} commands, these will automatically be printed on this slide as an overview of your presentation
\end{frame}

%----------------------------------------------------------------------------------------
%	PRESENTATION SLIDES
%----------------------------------------------------------------------------------------

\section{Motivation}
\subsection{Warum redundanz}
\begin{frame}
	\frametitle{Überblick}
	\tableofcontents[currentsubsection]
\end{frame}

\begin{frame}
	\frametitle{Redundanz}
	Warum:
	\begin{itemize}
			\item 1 
			\item 2
			\item 2
	\end{itemize}	

\end{frame}


\section{N-Version Programmierung}
\subsection{Definition}
%
%
\begin{frame}
	\frametitle{Definition}
	\enquote{\emph{N-version programming is defined as the independent generation of $ N \geq 2 $ functionally equivalent programs, called \enquote{versions}, from the same initial specification.}} \cite{Chen1978}

\end{frame}
%
\subsection{Konzepte}
%
\subsection{Annahmen}
%
\subsection{Beispiele}
%

\section{Experimente}
\subsection{Wie entscheidend ist die Spezifikation?}
\subsection{Unabhängigkeit der Fehler in Versionen}
\subsection{Aktuelle Forschungsprojekte}
\section{Fazit}
\begin{frame}
	\frametitle{Überblick}
	\tableofcontents[currentsection]
\end{frame}
%
\begin{frame}
	\frametitle{Fazit - Beobachtungen}

	\begin{itemize}
		\item N-Version Programmierung ermöglicht Fehlertoleranz eines Software-Systems durch funktionelle Redundanz
		\item Erhöhte Kosten durch Entwicklung mehrerer Versionen
		\item Potentiell niedrigere Kosten durch geringeren Testaufwand
		\item Unabhängigkeit der Fehler in Frage gestellt
		\item Einsatzbeispiele vor allem in der Luftfahrt
	\end{itemize}
\end{frame}
%
%
\begin{frame}
	\frametitle{Fazit - Ausblick}
	
	\begin{itemize}
		\item Neue Einsatzmöglichkeiten im Bereich Webservices
		\item Neues Ziel der Robustheit gegen böswillige Angriffe
		\item Kommunikationsverbot zwischen Entwicklerteams lockern
		\item Benötigte Kenntnisse der Programmierer überprüfen
	\end{itemize}
\end{frame}
%
%\input{literatur.tex}

\begin{frame}[allowframebreaks]
	\frametitle{Literatur}
	\bibliographystyle{apalike}
	\bibliography{literatur}
\end{frame}


\section{}
\begin{frame}
	\begin{center}
		\huge{Vielen Dank für die Aufmerksamkeit}
	\end{center}
	

\end{frame}


\end{document} 