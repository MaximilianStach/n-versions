\section{Fazit}
\begin{frame}
	\frametitle{Überblick}
	\tableofcontents[currentsection]
\end{frame}
%
\begin{frame}
	\frametitle{Fazit - Beobachtungen}

	\begin{itemize}
		\item N-Version Programmierung ermöglicht Fehlertoleranz in Software-Systemen durch funktionelle Redundanz
		\item Erhöhte Kosten durch Entwicklung mehrerer Versionen
		\item Ersparnisse beim Testaufwand
		\item Unabhängigkeit der Fehler in Frage gestellt
		\item Kombination von aufwendigen Tests und multiplen Versionen in sicherheitskritischen Systemen		
		\item Einsatzbeispiele vor allem in der Luftfahrt
	\end{itemize}
\end{frame}
%
%
\begin{frame}
	\frametitle{Fazit - Ausblick}
	
	\begin{itemize}
		\item Neue Einsatzmöglichkeiten im Bereich Webservices
		\item Neues Ziel der Robustheit gegen böswillige Angriffe
		\item Kommunikationsverbot zwischen Entwicklerteams lockern
		\item Benötigte Kenntnisse der Programmierer überprüfen
	\end{itemize}
\end{frame}
%