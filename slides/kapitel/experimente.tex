\section{Experimente}
%
\subsection{Wahl der Spezifikationssprache}
%
%
\begin{frame}
	\frametitle{Überblick}
	\tableofcontents[currentsubsection]
\end{frame}
%
\begin{frame}
	\frametitle{Experimente - Wahl der Spezifikationssprache (1)}
	Experiment von 1978-1983 \cite{Avizienis:1984:FTD:1319725.1320045}:
	\begin{itemize}
		\item Szenario: Flughafenverwaltung
		\item Spezifikation in OBJ, PDL und Englisch geschrieben
		\item 18 Versionen in PL/1 implementiert
		\item 7 OBJ, 5 PDL und 6 englische Versionen
		\item PDL-Spezifikation mit 74 Seiten vs. 13 bei der OBJ- und 10 bei der englischen Spezifikation
	\end{itemize}
	
\end{frame}
%
%
\begin{frame}
	\frametitle{Experimente - Wahl der Spezifikationssprache (2)}
	Durchführung des Experiments mit 100 Testtransaktionen:
	\begin{itemize}
		\item Transaktionsendpunkte als Vergleichsstellen
		\item 18-facher Vergleichsalgorithmus zur Ermittlung der korrekten Ergebnisse
		\item 816 Tripel-Kombinationen der Versionen getestet
	\end{itemize}	
\end{frame}
%
%
\begin{frame}
	\frametitle{Experimente - Wahl der Spezifikationssprache (3)}
	Beobachtungen:
	\begin{itemize}
	\item Reine PDL-Tripel leicht zuverlässiger
	\item Jedoch kein gravierender Unterschied bezüglich der Spezifikationssprachen
	\item Mangelnde Präzision und Ausdrucksmöglichkeiten der untersuchten Sprachen
	\item Unabhängige Fehlerursachen führten häufig zu den selben falschen Ergebnissen.
	\end{itemize}	
\end{frame}
%
%
\subsection{Unabhängigkeit der Fehler}
%
\begin{frame}
	\frametitle{Überblick}
	\tableofcontents[currentsubsection]
\end{frame}
%
\begin{frame}
	\frametitle{Experimente - Unabhängigkeit der Fehler?}
	\begin{itemize}
		\item Annahme der N-Version Programmierung:
			\begin{itemize}
				\item Unabhängige Entwicklungskonditionen führen zu unabhängigen Fehlern in den einzelnen Versionen
				\item Verschiedene Versionen versagen unabhängig voneinander
				\item Zuverlässigkeit des Gesamtsystems deutlich höher als die der einzelnen Versionen
			\end{itemize}
			\pause
			\item Aber:
			\begin{itemize}
				\item Programmierer tendieren bei anspruchsvollen Aufgaben dazu die selben Fehler zu machen
				\item Enorme zusätzliche Kosten in der Entwicklung multipler Versionen
				\item Lohnt sich der Aufwand?
				\item $\implies$ Annahme der Fehlerunabhängigkeit untersuchen
			\end{itemize}
		
	\end{itemize}
	
\end{frame}
%
%
\begin{frame}
	\frametitle{Experimente - Knight \& Leveson (1)}

	 Experiment zur Überprüfung der Unabhängigkeit von Fehlern in Versionen \cite{Knight:1986:EEA:10677.10688}:
		\begin{itemize}
			\item Raketenabwehrsystem als Zielprogramm
			\item 27 Versionen in Pascal auf Basis einer Spezifikation
			\item 241 Bit Vergleichsvektoren
			\item 1 Millionen automatisch generierte Testfälle
			\item Existierendes \enquote{\emph{Goldprogramm}} als Vergleichsmaßstab
		\end{itemize}

\end{frame}
%
%
\begin{frame}
	\frametitle{Experimente - Knight \& Leveson (2)}
	
	Ergebnisse:
	\begin{itemize}
		\item Hohe Zuverlässigkeit in allen Versionen ($\geq 99\%$)
		\item Trotzdem hohes Auftreten gemeinsamer Fehler in Testfällen
		\item In 1255 Fällen versagte mehr als eine Version
		\item In 2 Fällen versagten sogar 8 Versionen
		\item Mehr korrelierte Fehler als bei einer Normalverteilung bei Unabhängigkeit der Fehler anzunehmen
		\item Annahme der Fehlerunabhängigkeit zurückgewiesen
	\end{itemize}
	
\end{frame}

\subsection{Aktuelle Forschungsprojekte}
%
%
\begin{frame}
	\frametitle{Überblick}
	\tableofcontents[currentsubsection]
\end{frame}
%
\begin{frame}
	\frametitle{Aktuelle Forschungsprojekte}
	
	\center{\huge{Wie geht's weiter?}}
	
\end{frame}
%
%
\begin{frame}
	\frametitle{Cloud-basierte Antiviren-Software}
	
	CloudAV, Antiviren-System als Netzwerk-Service in der Cloud \cite{Oberheide:2008:CNA:1496711.1496718}: 
	\begin{itemize}
		\item 10 verschiedene Viren-Erkennungssysteme
		\item $35\%$ höhere Erkennungsrate als herkömmliche Systeme bei aktuellen Gefahren
	\end{itemize}
	\begin{figure}
		\includegraphics[scale=0.2]{grafiken/antivir.png}		
		\caption{Konzept CloudAV
			\footnotemark		
		}		
	\end{figure}
	\footnotetext{Quelle: \cite{Oberheide:2008:CNA:1496711.1496718}}
\end{frame}
%
%
\begin{frame}
	\frametitle{N-Version Website}

	\begin{itemize}
		\item 3 Versionen eines Online-Auktionshauses \cite{zero-day}
		\item Unterschiede in den Komponenten:
		\begin{itemize}
			\item Betriebssystem
			\item Webserver
			\item Programmiersprache
			\item DBMS
		\end{itemize}
		\item Ziel: Robustheit gegenüber Zero-Day-Exploits
		\item Schwachstelle: Verwaltender HTTP-Dispatcher

	\end{itemize}
	
\end{frame}
%