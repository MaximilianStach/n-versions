\section{N-Version Programmierung}
\subsection{Definition}
%
%
\begin{frame}
	\frametitle{Überblick}
	\tableofcontents[currentsubsection]
\end{frame}
%
%
\begin{frame}
	\frametitle{N-Version Programmierung - Definition}
	\begin{block}{Definition: N-Version Programmierung \cite{Chen1978}}
		\enquote{\emph{N-version programming is defined as the independent generation of $ N \geq 2 $ functionally equivalent programs, called \enquote{versions}, from the same initial specification.}}
	\end{block}
\end{frame}
%
\subsection{Konzepte}
%
%
\begin{frame}
	\frametitle{Überblick}
	\tableofcontents[currentsubsection]
\end{frame}
%
%
\begin{frame}
	\frametitle{N-Version Programmierung - Konzepte (1)}
	Nebenläufige Berechnung der Ergebnisse der N-Versionen:
	\begin{figure}
		\includegraphics[scale=0.3]{grafiken/multi-thread-n-version.png}		
		\caption{N-Version Software-Unit im Falle von Multi-Threading
			\footnotemark		
		}		
	\end{figure}
	\footnotetext{Quelle: \cite{lucent}}
\end{frame}
%
%
\begin{frame}
	\frametitle{N-Version Programmierung - Konzepte (2)}
	Besondere Komponenten einer N-Version Software-Unit
	\begin{itemize}
		\item Vergleichsvektoren
		\begin{itemize}
			\item Zustand der jeweiligen Version
			\item Variablen, Ereignisse...
		\end{itemize}
		\pause
		\item Vergleichsindikatoren
		\begin{itemize}
			\item Ausgang eines Vergleichs
			\item Anzustoßende Aktion
		\end{itemize}
		\pause
		\item Synchronisationsmechanismen
		\begin{itemize}
			\item Signale zwischen Versionen und Treiber
		\end{itemize}
	\end{itemize}
\end{frame}
%
%
\begin{frame}
	\frametitle{N-Version Programmierung - Konzepte (3)}
	Aufstellen der gemeinsamen Spezifikation:
	\begin{itemize}
		\item Zu implementierende Funktion
		\item Datenformat der Vergleichsvektoren
		\item Zu verwendende Vergleichsalgorithmus
		\item Auf Ausgänge der Vergleiche folgende Aktionen
	\end{itemize}
\end{frame}
%
%
\subsection{Beispiele}
%
